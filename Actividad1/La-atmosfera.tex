
\documentclass[a4paper,12pt]{report}
\usepackage{graphicx}
\usepackage[spanish]{babel}
\usepackage[utf8]{inputenc}
\usepackage[T1]{fontenc}
\usepackage[a4paper,top=3cm,bottom=2cm,left=3cm,right=3cm,marginparwidth=1.75cm]{geometry}

\begin{document}

\begin{titlepage}
\begin{center}
\vspace{0.5cm}
\includegraphics[width=6.5cm]{unison.jpg}
\\[0.5cm]
{\fontfamily{arial}\fontsize{28}{6}\selectfont{Universidad de Sonora}}
\\[0.5cm]
{\fontfamily{arial}\fontsize{22}{6}\selectfont{Departamento de Física}}
\\[2cm]
{\fontfamily{arial}\fontsize{24}{6}\selectfont{La estructura de la atmósfera}}
\\[2cm]
{\fontfamily{arial}\fontsize{22}{6}\selectfont{Alumno:}}
\\[0.5cm]
{\fontfamily{arial}\fontsize{18}{6}\selectfont{Torres Flores Luis Alfonso}}
\\[2cm]
{\fontfamily{arial}\fontsize{22}{6}\selectfont{Profesor:}}
\\[0.5cm]
{\fontfamily{arial}\fontsize{18}{6}\selectfont{Dr. Carlos Lizarraga Celaya}}
\\[3cm]
{\fontfamily{arial}\fontsize{14}{6}\selectfont{Miércoles 25 de Enero del 2017}}
\end{center}
\end{titlepage}

{\fontfamily{arial}\fontsize{24}{6}\selectfont{Breve resumen}}
\\[0.5cm] 
 Este trabajo hablara sobre la atmosfera, en parte de su composición y sobre las distintas capas de que la componen, mencionando también la variación que sufre cada una de estas en su presión y temperatura conforme tienen una mayor altura y los elementos que se encuentran en dichas partes.
\\[2cm]

{\fontfamily{arial}\fontsize{24}{6}\selectfont{Introducción}}
\\[0.5cm]
La atmosfera no es solo una gran capa, esta tiene distintas partes que lo conforman y nos brindan la seguridad en la que hoy en día disfrutamos. La vida se forma en la superficie, la zona con la menor altura en todo lo que conforma la atmosfera, en este lugar tenemos todos los recursos necesarios para sobrevivir, desde los climas perfectos para agricultura o cuando menos suficientes para desarrollarse, como los elementos necesarios en el aire para poder respirar. 

Esta capa nos protege de los rayos ultravioletas que emite nuestro sol, dejando pasar solo una parte de ellos los cuales calientan nuestro planeta y salen nuevamente. En esto dicho intervienen numerosos elementos que existen en las capas de las que hablaremos. La atmosfera si bien hablaremos un poco de ella como si fuera una sola entidad, hablaremos sobre las capas que lo componen, a que alturas se encuentran y sus características como lo son temperatura y presión sin olvidarnos de que están formadas. 
\\[2cm]
\newpage

\tableofcontents

\chapter{atmosfera}
\section{Atmosfera}
La atmosfera de nuestro planeta Tierra tiene una altura de hasta 10,000 kilómetros, aunque la concentración de masa esta algo dispareja, puesto que más de la mitad esta en apenas los primeros seis kilómetros y el 75 por ciento en los primeros 11 kilómetros. La masa de la atmosfera es de $5.1x10^18$ kilogramos.

Nos encontramos con la capa de ozono, la cual se encuentra desde los 15 hasta los 50 kilómetros de altitud.  Dicha capa posee el 90 por ciento del ozono presente en la atmosfera y absorbe el 97 al 99 por ciento de la radiación ultravioleta que emite el sol.

La atmosfera se compone de diversos gases, siendo estos, junto con sus respectivas proporciones:

\begin{itemize}
\item Nitrógeno: constituye el 78 por ciento del volumen del aire. 

\item Oxígeno: Representa el 21 por ciento del volumen del aire. Es un componente primordial dado que la mayoria de los seres vivos lo necesitan para poder vivir.

\item Otros gases: El más abundante es el argon, el cual conforma el 0.9 por ciento del volumen del aire. Al ser un gas ideal no reacciona.

\item Dióxido de carbono: Representa el 0,03 por ciento del volumen del aire. Las plantas lo necesitan para realizar la fotosíntesis, y es el residuo de la respiración y de las reacciones de combustión. Este gas junto con el vapor de agua funcionan como refrigerante, pues ayudan a mantener en una temperatura aceptable donde se pueda desarrollar la vida.

\item Partículas sólidas y líquidas: Estos materiales tienen una distribución muy variable, dependiendo de los vientos y de la actividad humana. Entre los líquidos, la sustancia más importante es el agua en suspensión que se encuentra en las nubes.

\end{itemize}

Como podemos apreciar, el nitrógeno es el elemento que más abunda en la atmosfera, encontrándose junto al oxígeno en forma biatómica, es decir, que son dos átomos del mismo elemento unidos. El nitrógeno es un gas inerte, es decir que no reacciona, a diferencia del oxígeno que es un gas altamente reactivo.


\section{Capas de la atmosfera}

La atmosfera se compone de diversas capas, podemos destacar cinco de ellas. Cada una mantiene composiciones algo variables, son afectadas de distinta manera por la temperatura y la presión debido a la altura en la que se encuentran. Detallaremos un poco cada una de ellas.


\subsection{Toposfera}

En esta capa se desarrolla la vida. Esta capa tiene apenas unos ocho kilómetros de espesos en los polos y 16 kilómetros en el ecuador. La temperatura desciende a razón de seis grados por kilómetro. Dado que en esta capa es en la que se desarrolla la vida y como ya se había mencionado antes, aquí se encuentra el 75 por ciento de la masa de la atmosfera.

\subsection{Estratosfera}

Esta capa toma lugar desde los diez hasta los 50 kilómetros de altitud. Los gases comienzan a separarse y agruparse en formas de capas conforme sea su peso. Debido a esto podemos notar que es en esta parte donde se encuentra la capa de ozono. El oxígeno y anhídrido carbónico son altamente escazas y aumenta la proporción de hidrógeno. Actúa como regulador de la temperatura, siendo en su parte inferior cercana a los -60 grados Celsius y aumentando con la altura hasta los 10 o 17 grados Celsius.

\subsection{Mesosfera}

En esta capa las temperaturas pueden descender hasta los 70 grados bajo cero conforme aumenta la altitud en la que se encuentra. A partir de la zona donde termina la estratosfera inicia esta capa de nombre mesosfera, la cual se extiende hasta los 80 kilómetros de altura, con una temperatura que descenderá hasta los 80 o 90 grados bajo cero dada su composición gaseosa tan reducida de masa.

\subsection{Termosfera}

Se encuentra entre los 90 y 400 kilómetros de altura. Esta capa está formada de iones, átomos cargados eléctricamente, lo que la convierte en la principal causa de que sea posible las transmisiones de radio y televisión por su propiedad de reflejar ondas electromagnéticas. Se compone principalmente de nitrógeno y es la capa que fragmenta a los meteoritos u otros objetos por la fricción que se genera aquí. A diferencia de las capas anteriores esta puede pasar desde 73 grados bajo cero hasta los 1500 grados al estar por encima de la capa de ozono. 

\subsection{Exosfera}
Esta es la capa más externa de la atmosfera. Se localiza hasta los 580 kilómetros de altitud y se extiende hasta los 10,000 kilómetros. Aquí la fuerza gravitacional es despreciable, por lo que el gas puede escapar, es la zona que únicamente se delimita por el campo magnético causado por la ionización de los escasos gases que se encuentran aquí pues gradualmente disminuyen y se asemejan al vacío del espacio exterior.

Aquí podemos encontrar satélites artificiales que han sido enviados para estudiar el clima como puede ser un ejemplo. Hablar de temperatura es algo completamente distinto a las demás capas, pues la ausencia de gases hace que este lugar, como ya se mencionó, se asimile al espacio exterior por lo que está completamente expuesto a rayos ultravioletas que provienen del sol. Algunos gases que pueden encontrarse aquí son hidrogeno, helio dióxido de carbono y oxigeno atómico.


\newpage

\chapter{Conclusión}
La atmosfera es algo inmensamente esencial para que la vida ocurra en el planeta. Con sus capas que nos protegen de algún fenómeno como son los rayos ultravioletas, permiten pasar cierta cantidad de calor, ayuda a destruir meteoritos y más que nada en la superficie se compone de los elementos esenciales para la vida como la vemos en nuestro planeta. Estamos hablando de algo altamente complejo dado que aquí suceden los fenómenos meteorológicos que aún se estudian.

Entender la atmosfera es algo muy importante, pues si podemos comprender en su totalidad todo lo que sucede en estas capas podremos ya sea utilizarlas a conveniencia o entender con mayor profundidad la protección que nos da, lo que nos brinda y su situación como lo sería la degradación de la capa de ozono y la alta contaminación generada por las ciudades. Dada esta necesidad la humanidad se ha visto forzada a crear métodos y herramientas para su estudio, su comprensión de no solo el aire en la superficie o en capas más altas sino de mis mismos fenómenos, con lo que hemos logrado obtener la actual información disponible de la atmosfera.


\chapter{Bibliografía empleada}

|1|CLIMATE. NC State University. Visitado el 25 de Enero de 2017. http://climate.ncsu.edu/edu/k12/.AtmStructure
\\[1cm]
|2|Windows to the Universe. Visitado el 25 de Enero de 2017. http://www.windows2universe.org/earth/Atmosphere/overview.htm\&lang=sp
\\[1cm]
|3|BIOCAB. Biologo Nasif Nahle Sabag. Visitado el 26 de Enero de 2017. http://www.biocab.org/Densidad-Energia-Atmosfera.html
\\[1cm]
|4|GeoEnciclopedia. Visitado el 26 de Enero de 2017. http://www.geoenciclopedia.com/capas-de-la-atmosfera/


\end{document}
