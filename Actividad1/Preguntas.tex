\documentclass[a4paper,12pt]{report}
\usepackage{graphicx}
\usepackage[spanish]{babel}
\usepackage[utf8]{inputenc}
\usepackage[T1]{fontenc}
\usepackage[a4paper,top=3cm,bottom=2cm,left=3cm,right=3cm,marginparwidth=1.75cm]{geometry}

\begin{document}

\begin{itemize}

\item ¿Cual es tu primera impresión de uso de LaTeX?

Es un editor de textos bastante complejo cuando alguien recién comienza a utilizarlo pero muy completo, dificulta mucho las cosas pero me parece que acostumbrándose a los comando y trabajando más con él se puede facilitar enormemente todo.

\item ¿Qué aspectos te gustaron más?

Que fuera bastante automático en algunos aspectos, como lo es hacer una bibliografía o crear el índice que en otro editor de textos como Word habría sido algo más tardado, además que se ve mejor a la vista.

\item ¿Qué no pudiste hacer en LaTeX?

Francamente, muchísimas cosas por lo nuevo que me es trabajar en LaTeX. Algunas de ellas son insertar imágenes sin tardar una gran cantidad de tiempo arreglando los problemas, darle estilo propios a las letras sin que me saltaran muchos errores, darles color, entre otros.

\item tu experiencia, comparado con otros editores, ¿cómo se compara LaTeX? 

Mucho más profesional, detallado, mejor a la vista, automatizado y a la larga mucho mejor, pero actualmente demasiado complicado que es difícil hasta hacer una portada con imágenes y algunos detalles.

\item ¿Qué es lo que mas te llamó la atención en el desarrollo de esta actividad?

Lo complicado que era, lo enormemente complicado que es manejar LaTeX sin conocimientos previos.

\item ¿Qué cambiarías en esta actividad?

Realmente nada, sirve como una práctica antes de hacer trabajos más complejos y acostumbrarse al editor.

\item ¿Que consideras que falta en esta actividad? 

Unas cuantas guías más podría ser o algunas referencias extras. Aunque fuera de eso me parece que está bien, como mencione anteriormente, sirve como una pequeña practica y algo como esto da paso a buscar guías, formas de utilizar comandos, como manejarlos, entre otras cualidades de LaTeX.

\item ¿Puedes compartir alguna referencia nueva que consideras util y no se haya contemplado? 

Realmente no, observe mucho otros archivos para darme una idea de cómo funcionaban y leer las guías recomendadas en la actividad misma.

\item ¿Algún comentario adicional que desees compartir? 

LaTeX es muy complicado manejarlo al inicio, pero se nota rápidamente su utilidad y cuánto debe facilitar el trabajo posteriormente.

\end{itemize}

\end{document}
