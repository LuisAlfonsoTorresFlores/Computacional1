 \documentclass[a4paper,12pt]{article} 

%%%%%%%%%%%%%%%%%%%%%%%%%%%%%%%%%%%%%%%%%%%%%%%%%%%%%%%%%%%%%%%%%%%%%%%%%%%%
\usepackage[utf8]{inputenc}
\usepackage[spanish]{babel}
\usepackage{amsmath}
\usepackage{amsfonts}
\usepackage{amssymb} 
\usepackage{graphicx} 
\usepackage{hyperref} 
\usepackage{wrapfig}
\usepackage{enumitem}
\usepackage{blindtext}
\usepackage{fancyhdr}
\usepackage{float}
\usepackage{eurosym}
\usepackage{color}
\usepackage{titling}
\usepackage{amssymb, amsmath, amsbsy} % simbolitos
 \usepackage{upgreek} % para poner letras griegas sin cursiva
 \usepackage{cancel} % para tachar
 \usepackage{mathdots} % para el comando \iddots
 \usepackage{mathrsfs} % para formato de letra
\usepackage{stackrel} % para el comando \stackbin
\usepackage{lipsum}
\usepackage{tocbibind}
\usepackage[T1]{fontenc}
\usepackage[left=3cm,right=3cm,top=3cm,bottom=4cm]{geometry}
\pagestyle{fancy}
%%%%%%%%%%%%%%%%%%%%%% CABECERAS %%%%%%%%%%%%%%%%%%%%%%%%%%%%%%%%%%%%%%%%%%%
\newcommand{\hsp}{\hspace{20pt}}
\newcommand{\HRule}{\rule{\linewidth}{0.5mm}}
\headheight=50pt
\newcommand{\vacio}{\textcolor{white}{holacaracola}}
%%% NUMERACION DE ECUACIONES
\renewcommand{\theequation}{\thesection.\arabic{equation}}
% COLOR AZUL PARA TEXTOS EN PORTADA
\definecolor{azulportada}{rgb}{0.16, 0.32, 0.75}
% Azul para textos de headings
\definecolor{azulinterior}{rgb}{0.0, 0.2, 0.6}

%%%%%%%%%%%%%%%%%%% DATOS DEL PROYECTO %%%%%%%%%%%%%%%%%%%%%%%%%%%%%%%%%%%%%
\title{Limpieza y preparación de datos usando Emacs}
\author{}
\newcommand{\director}{Carlos Lizárraga Celaya}

\begin{document}
\begin{titlepage}
\begin{center}
\vspace{1cm}

\includegraphics[width=5.5cm]{unison.jpg}
\\[0.5cm]
{\fontfamily{phv}\fontsize{24}{6}\selectfont{UNIVERSIDAD DE SONORA}}\\
[1em]
{\fontfamily{phv}\fontsize{16}{5}\selectfont{DEPARTAMENTO DE FÍSICA}}\\
[4em]
\textcolor{azulportada}
{\fontfamily{phv}\fontsize{30}{5}\selectfont{\textsc{\thetitle}}}\\
% Autor del trabajo de investigación
[1cm]
{\fontfamily{phv}\fontsize{16}{5}\selectfont{Alumno:}}\\
[0.2cm]
%Equipo sfdsfshkfhsfhsjfs
{\fontfamily{phv}\fontsize{14}{5}\selectfont{Luis Alfonso Torres Flores}}\\
[1cm]
%{\Huge\textbf{\thetitle}}\\
{\fontfamily{phv}\fontsize{16}{5}\selectfont{Profesor}}\\
[0.2cm]
{\fontfamily{phv}\fontsize{16}{5}\selectfont{\director}}\\
[4.5cm]
{\fontfamily{phv}\fontsize{14}{5}\selectfont{07 de Febrero de 2017}}\\
[4cm]
\end{center}
\restoregeometry
\end{titlepage}

\newpage
%%%Encabezamiento y pie de página
\renewcommand{\headrulewidth}{0.5pt}
\fancyhead[R]{
	\textcolor{azulinterior}{\fontfamily{phv}\fontsize{14}{4}\selectfont{\textbf{\thetitle}}}\\
\textcolor{azulportada}{\fontfamily{phv}\fontsize{10}{3}\selectfont{Curso de Introducción a la Física Moderna I}}\\
{\fontfamily{phv}\fontsize{10}{3}\selectfont{\theauthor}}}
\fancyhead[L]{\vacio}

\newpage
\tableofcontents
\newpage

%----------------------------------------------------------------------------------------
\section{Resumen}
Emacs es una buena herramienta para lo que ordenar datos se refiere. Como hemos estado haciendo en clase, trabajamos con listas de datos de una cantidad enorme, lo cual se nos dificulta trabajar con ellos. En este trabajo hablaremos solamente un poco sobre el uso que se dio Emacs a la actividad número 2 y de la forma que nos ayudó al ordenar y conseguir los datos.
%----------------------------------------------------------------------------------------
\section{Introducción}
\noindent
Emacs, un programa que para este caso utilizamos para limpiar nuestros datos y obtener a los mismos. Gracias a un script proporcionado en la misma actividad se consiguieron los datos de todo 1 año, en este caso estamos hablando de un año entero de los datos atmosféricos de Brownsville. Limpiamos los datos de todo elemento que podría complicarnos de graficarlos por lo que nos resultaría al final mucho más fácil trabajar con ellos, como sería graficarlos. Es una herramienta realmente útil que puede realizar cambios que coincidan en algunos caracteres y reemplazarlos por otros. Hablaremos un poco de la manera que lo utilizamos en este caso, como pudimos obtener los datos con los cuales trabajamos, el proceso por el cual se pasó.\\

Hubo hasta cuatro horas, las cuales fueron 12Z, 00Z, 18Z y 06Z, considerando que cada día tenía una enorme cantidad de mediciones tratar de manejarlos manualmente, es cuando menos, muy tardado, requiere de muchísimo tiempo, pero un solo comando con repetidos usos, fue capaz de reducir el tiempo requerido enormemente.
\newpage
%-----------------------------------------------------------------------------------------
\section{Datos en Emacs}

%----------------------------------------------------------------------------
\subsection{Sondeos}
\noindent
A continuación, se muestra una tabla de datos, donde son el número de mediciones que hicieron en sus correspondientes meses en dos horas, las 12Z y las 00Z.

\begin{center}
\begin{tabular}{| l | c | r |}
\hline
Mes & 12Z & 00Z \\ \hline
Enero & 31 & 31 \\ \hline 
Febrero & 28 & 28 \\ \hline
Marzo & 31 & 31 \\ \hline
Abril & 30 & 30 \\ \hline
Mayo & 31 & 30 \\ \hline
Junio & 30 & 30 \\ \hline
Julio & 31 & 31 \\ \hline
Agosto & 31 & 31 \\ \hline 
Septiembre & 30 & 30\\ \hline 
Octubre & 31 & 31 \\ \hline
Noviembre & 30 & 30 \\ \hline
Diciembre & 31 & 31 \\ \hline

\hline
\end{tabular}
\end{center}

\noindent
Como podemos apreciar, se puede decir que, a diario, casi sin falta, se realizaron mediciones en sus correspondientes horas, lo que nos indica que se están manejando una enorme cantidad de datos
%--------------------------------------------------------------------------------------
\subsection{Graficas}
\noindent
La grafica que vemos debajo nos permite observar la presión con respecto a la altura.

\begin{center}
\includegraphics[width=11cm]{Grafica_de_Altura-Presion.png}
\end{center}
\noindent
Como podemos observar, la presión va disminuyendo conforme la altura aumenta. Por la forma que refleja la gráfica podemos asimilarla a una exponencial, lo cual concuerda con la ecuación para la presión del aire al variar la altura de manera considerable, la cual está dada por una exponencial. No podemos decir que toma un valor cero, pero si es muy cercano a cero, por lo que se aprecia que toca el eje Y.\\

\noindent
A continuación, observamos una gráfica de la temperatura contra la altura.

\begin{center}
\includegraphics[width=11cm]{Grafica_Altura-Temperatura.png}
\end{center}

\noindent
Notamos una variedad en los valores de la temperatura, pero debemos de recordar a la primera actividad. Las capas dado los gases los cuales la conforman, varían de temperatura, por lo que no es de extrañar que dichos valores resultaran de esta manera. Podemos apreciar los cambios que se efectúan a manera de fases, por lo que podríamos tratar de teorizar por cual capa de la atmosfera se habría encontrado el globo en esos puntos dada la característica de temperatura que muestran en sus respectivas alturas.
%----------------------------------------------------------------------------------------
\newpage
\section{Obtención de datos}
\noindent
La manera de obtener datos fue simple. Se utilizó el script proporcionado y tras algunas modificaciones se dispuso a usarlo para poder conseguir los datos de todo un año en la página correspondiente. Al descargar cada mes se tomaba intervalos de cinco segundos, además que se tuvo en cuenta que habría que tomar turnos para no sobrecargar el servidor de la página por lo que había tomado un tiempo.

\subsection{Script}
\# Descarga por mes. Cambiar año de consulta. Ajustar el numero de estacion.

\#!/bin/bash

 

\# Despues de editar: chmod 755 script1.sh

\# Para ejecutar: ./script1.sh

 

IFS=":"

LISTM31="01:03:05:07:08:10:12"

\#LISTM31="01:03:05:07"

LISTM30="04:06:09:11"

\#LISTM30="04:06"

LISTM28="02"

 

\# Script para bajar info por mes. Año 2016, dentro del URL:  YEAR=2015

\# Months 31 days

for i in $LISTM31 ; do
    /usr/bin/wget "http://weather.uwyo.edu/cgi-bin/sounding?region=naconf\&TYPE=TEXT\%3ALIST\&YEAR=2016\&MONTH=\$i\&FROM=0100\&TO=3112\&STNM=72250"
       /bin/sleep 5
\\
done
\\
\# Months 30 days
\\
for i in \$LISTM30 ; do
    /usr/bin/wget "http://weather.uwyo.edu/cgi-bin/sounding?region=naconf\&TYPE=TEXT\%3ALIST\&YEAR=2016\&MONTH=\$i\&FROM=0100\&TO=3012\&STNM=72250"
       /bin/sleep 5
\\
done
\\
\# Feb. 28 days
\\
for i in \$LISTM28 ; do
\\
    /usr/bin/wget "http://weather.uwyo.edu/cgi-bin/sounding?region=naconf\&TYPE=TEXT\%3ALIST\&YEAR=2016\&MONTH=\$i\&FROM=0100\&TO=2812\&STNM=72250"
       /bin/sleep 5
\\
done$
\end{document}
\end{document}
